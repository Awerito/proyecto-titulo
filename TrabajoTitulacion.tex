\documentclass[letter,12pt]{report}
\usepackage[utf8]{inputenc}
\usepackage[T1]{fontenc}
\usepackage[spanish, es-tabla]{babel}
\usepackage[sfdefault, condensed]{roboto}
\usepackage[margin=1cm]{geometry}
\usepackage{multicol,graphicx,fancyhdr,eso-pic,url,float,cite,lmodern,listings,times,textcomp, amsthm,amsmath,amssymb,dsfont,color,colortbl,sidecap,xspace,epic,eepic,anysize,setspace, hyperref, pdflscape,lscape}
\usepackage{blindtext}

%% Contenido
% - Titulo
% - Objetivo General
% - Definiciones
% - Estado del Arte (No va)
%% Contenido

%%%%%%Glosario
%\usepackage[acronym]{glossaries}
%\makeglossaries
%\renewcommand{\glossaryname}{Glosario}
%\renewcommand{\acronymname}{Acrónimos}


%\usepackage{apacite} %bibliografias Bibtex

%Tipos de Letra
%\renewcommand{\rmdefault}{phv} % Arial
\usepackage{mathptmx} %Times
%Margenes
\marginsize{3cm}{2cm}{2cm}{2cm}
\spacing{1}%interlineado

  \providecommand{\keywords}[1]{\textbf{\textit{Palabras Clave---}} #1}

\newcommand\BackgroundPic{ \put(-3,0){ \parbox[b][\paperheight]{\paperwidth}{ \vfill \centering \includegraphics[width=\paperwidth,height=\paperheight]{portada.jpg} \vfill }}} 
  
  %Definicion de Colores
\definecolor{gray97}{gray}{.97}
\definecolor{gray75}{gray}{.75}
\definecolor{gray45}{gray}{.45}
\definecolor{verdeo}{rgb}{0,.5,0.2}
\definecolor{listinggray}{gray}{0.9}
\definecolor{lbcolor}{rgb}{0.9,0.9,0.9}
\newcommand\gris[1]{\textcolor[gray]{.35}{\emph{#1}}}
\newcommand\rojo[1]{\textcolor[rgb]{1,0,0}{#1}}
\newcommand\blue[1]{\textcolor[rgb]{0,0,1}{{#1}}}
\newcommand\azul[1]{\textcolor[rgb]{0,0,1}{#1}}
\newcommand\verde[1]{\textcolor[rgb]{0,.5,0.2}{#1}}
\newcommand\naranjo[1]{\textcolor[rgb]{1.00,0.36,0.06}{\textbf{#1}}}
\newcommand\blanco[1]{\textcolor[rgb]{1,1,1}{\textbf{#1}}}
\newcommand {\red}[1]{\textcolor[rgb]{1.00,0.00,0.00}{#1}}
\newcommand\cita[1]{{\scriptsize \begin{flushright}\emph{(#1)}\end{flushright}}}
  

  
%%%Entornos de desarrollo
\newtheorem{ejemplo}{Ejemplo}
\newtheorem{definir}{Definición}
\newtheorem{prueba}{Prueba}
\newtheorem{demo}{Demostración}
\newtheorem{obs}{Observación}

\newcommand{\ignore}[1]{}
  
  %%%CODIGOS DE PROGRAMACION
\lstset{%backgroundcolor=\color{lbcolor},
	frame=Ltb, framerule=0pt, aboveskip=0.5cm, tabsize=4, rulecolor=, language=C, %%%CAMBIAR POR LENGUAJE DE PREFERENCIA
 stringstyle=\ttfamily,  %basicstyle=\footnotesize,
        upquote=true, aboveskip={1.5\baselineskip}, columns=fixed, showstringspaces=false, extendedchars=true,breaklines=true, prebreak = \raisebox{0ex}[0ex][0ex]{\ensuremath{\hookleftarrow}}, showtabs=false, showspaces=false, showstringspaces=false,
        %tipos de letra y colores
        identifierstyle=\ttfamily,
        keywordstyle=\bfseries  \color[RGB]{0,2,216}, %palabras reservadas
        commentstyle= \scriptsize\color[rgb]{0,.5,0.2}, %comentarios
        stringstyle=\color[RGB]{216,0,114},%cadena de texto
        %numeracion de lineas
        framextopmargin=3pt, framexbottommargin=3pt, framexleftmargin=0.4cm,
        framesep=0pt, rulesep=.4pt, rulesepcolor=\color{black}, numbers=left, numbersep=15pt, numberstyle=\tiny, numberfirstline = false, breaklines=true,literate={á}{{\'a}}1 {é}{{\'e}}1 {í}{{\'i}}1 {ó}{{\'o}}1 {ú}{{\'u}}1
  {Á}{{\'A}}1 {É}{{\'E}}1 {Í}{{\'I}}1 {Ó}{{\'O}}1 {Ú}{{\'U}}1
  {à}{{\`a}}1 {è}{{\`e}}1 {ì}{{\`i}}1 {ò}{{\`o}}1 {ù}{{\`u}}1
  {À}{{\`A}}1 {È}{{\'E}}1 {Ì}{{\`I}}1 {Ò}{{\`O}}1 {Ù}{{\`U}}1
  {ä}{{\"a}}1 {ë}{{\"e}}1 {ï}{{\"i}}1 {ö}{{\"o}}1 {ü}{{\"u}}1
  {Ä}{{\"A}}1 {Ë}{{\"E}}1 {Ï}{{\"I}}1 {Ö}{{\"O}}1 {Ü}{{\"U}}1
  {â}{{\^a}}1 {ê}{{\^e}}1 {î}{{\^i}}1 {ô}{{\^o}}1 {û}{{\^u}}1
  {Â}{{\^A}}1 {Ê}{{\^E}}1 {Î}{{\^I}}1 {Ô}{{\^O}}1 {Û}{{\^U}}1
  {œ}{{\oe}}1 {Œ}{{\OE}}1 {æ}{{\ae}}1 {Æ}{{\AE}}1 {ß}{{\ss}}1
  {ű}{{\H{u}}}1 {Ű}{{\H{U}}}1 {ő}{{\H{o}}}1 {Ő}{{\H{O}}}1
  {ç}{{\c c}}1 {Ç}{{\c C}}1 {ø}{{\o}}1 {å}{{\r a}}1 {Å}{{\r A}}1
  {€}{{\EUR}}1 {£}{{\pounds}}1 {Ñ}{{\~N}}1 {ñ}{{\~n}}1 {¿}{{?`}}1
}
%%%%FIN CODIGOS DE PROGRAMACION
\def\figurename{}
  
%%%%%%%%%%ENCABEZADO Y PIE DE PAGINA
%encabezado de las paginas pares e impares.
\rhead[PP]{Ingeniería Civil en Informática}
\renewcommand{\headrulewidth}{0.5pt}
%pie de pagina de las paginas pares e impares.
\lfoot[nombre]{Muñoz Viveros}
\rfoot[rut]{Universidad de los Lagos}
\renewcommand{\footrulewidth}{0.5pt}
%encabezado y pie de pagina de la pagina inicial de un capitulo.
\fancypagestyle{plain}{
\fancyhead[R]{Ingeniería Civil en Informática}
\fancyfoot[L]{Muñoz Viveros}
\fancyfoot[R]{Universidad de los Lagos}
\renewcommand{\headrulewidth}{0.5pt}
\renewcommand{\footrulewidth}{0.5pt}
}
\pagestyle{fancy} 
%%%%%%%%%%FIN ENCABEZADO Y PIE DE PAGINA 
 


\begin{document}

%%%%%%%%%%%PORTADA%%%%%%%%%%%%%%%%%%%%%
\setlength{\unitlength}{1 cm} %Especificar unidad de trabajo
\thispagestyle{empty}

\AddToShipoutPicture*{\BackgroundPic}

   \title{\scshape\Huge{Detección de presencia de parásitos en examen
parasitológico seriado de deposiciones con visión por computadora}\\\vspace{1cm}
        \Large Departamento de Ciencias de la Ingeniería\\
        \Large Ingeniería Civil en Informática\\
        \large \rojo{Sede Chinquihue km 6}, Chile}
   \author{
      Diego Ignacio Muñoz Viveros\\
      diegoignacio.munoz@alumnos.ulagos.cl
   }
   \date{Profesor Guía: Joel Sebastian Torres Carrasco\\  Co-guía: Carlos Dupré Alvarado\\
   \today}
   \maketitle
   \ClearShipoutPicture


\cleardoublepage
\pagenumbering{roman}
\setcounter{page}{1}

%%%%Agradecimientos
\input{Agradecimientos}
%%%%%%%%RESUMEN
\begin{abstract}
    El examen parasitológico seriado de deposiciones es un examen realizado para la
    detección de presencia parasitológica en pacientes. El examen se lleva a cabo con la
    toma de muestras de deposiciones del paciente y posteriormente se hace observación de
    las mismas por medio de microscopio, en la realización de la observación se documenta
    la confirmación y clasificación de presencia parasitológica en orden de dar con un
    tratamiento certero, eficaz y acorde a las detecciones.

    El uso de la visión por computadora espera demostrar el aumento en la precisión y
    velocidad de la detección y clasificación de parásitos con objetivo de reducir
    incertidumbre y error humano introducido en la observación manual ejercida en el
    proceso de observación del examen parasitológico seriado de deposiciones. La visión
    por computadora es la combinación de tecnologías que permite a las computadoras
    generar inferencia respecto a imágenes estáticas o en movimiento emulando la visión
    humana por medio de un aprendizaje basado en conjuntos de datos utilizados como
    ejemplos iniciales de los cuales el modelo generado ajustará sus parámetros.

    En el presente trabajo, se realizó una búsqueda de conjuntos de datos sobre parásitos,
    se desarrolló un modelo de clasificación de imágenes basados en visión por computadora y
    se probaron diferentes alternativas para mejorar la calidad de la identificación y la 
    clasificación.

    Finalmente, este trabajo aporta un marco experimental sobre un conjunto de datos acotado
    de parásitos, que puede ser liberado a producción, ampliando la cantidad de parásitos a
    identificar.

%Resume en un (1) párrafo el contenido del informe en un máximo de 350 palabras.
%Debe ser preciso:
%\begin{itemize}\justifying
  %\item Establece el problema
  %\item Dice porqué es interesante
  %\item Señala los logros y desafios
%\end{itemize}
%Un resumen debe ser llamativo, motivador, descriptivo y sin contenido específico. \textbf{No incluye}: citas, referencias, conclusiones, figuras ni tablas.
%
%
\keywords{Parasitólogia, Serializado, Deposiciones, Visión por Computadora, Modelo,
Conjunto de Datos, \textit{Underfiting}, \textit{Overfiting}}
\end{abstract}



%%%%%%INDICES
\tableofcontents
\listoffigures
\renewcommand{\listtablename}{Índice de tablas}
\listoftables
\renewcommand{\lstlistlistingname}{Índice de algoritmos}
\lstlistoflistings
%\addcontentsline
%%%%%%%%%%%%%FIN PORTADA%%%%%%%%%%%%%%%%










\cleardoublepage
\pagenumbering{arabic}
\setcounter{page}{1}






%%%%%%%%COMIENZO



\chapter{Formulación del Proyecto}\label{formulacion}
Es lo que comúnmente se conoce como introducción, conduce al lector desde un tema de un área general hacia un campo de investigación específico, describe el contexto, el problema, motiva al lector.

Introduce la terminología, destaca las contribuciones del documento y da una breve descripción de la organización de éste.

Ejemplo de uso de una referencia \cite{c02}. Ejemplo de referencia doble \cite{c02}.


Se debe describir en mayor detalle lo que se expone en el resumen (es válido tomar párrafos del resumen y exponerlos en forma detallada).

Exposición general del problema, señalando los enfoques y procedimientos actualmente en uso, así como la fundamentación o justificación del proyecto (existencia de problemas), mención a los objetivos generales y/o específicos (no es necesario que aparezcan los títulos: Objetivo General y Objetivo Específico, sino que se mencionen como parte natural de la descripción), propuesta de solución. Si es necesario (normalmente es muy útil) incluir un diagrama en que se visualiza la arquitectura del sistema y resultados esperados (del punto de vista del producto final).

Igual que en el caso anterior se recomienda comenzar con la misma frase del párrafo anterior referida al objetivo general, después en algún punto de la explicación de la propuesta de solución hacer mención implícita a los contenidos de los objetivos específicos (sin mencionar que son objetivos específicos). Describir además las mejoras a las que la solución conlleva (resultados esperados).


\section{Propósito}
Contextualiza el trabajo respecto de investigaciones previas de otros autores y propias, señala las diferencias con trabajos previos. Algunas veces se incluye en la introducción o bien en la discusión del trabajo (secciones finales). Largo aproximado: 2 páginas.

\section{Marco Teórico}
\subsection{Parasitología}
La parasitología \cite{c02} es la rama de las ciencias biologicas dedicada a el estudio de
organismos, denominados parásitos, que dependen de otro para poder sobrevivir y que
ocasionan grandes daños a las especies de las cuales dependen, relación llamada
parasitismo.

La parasitología es una disciplina con aplicación en campos variados como medicina,
farmacologia y veterinaria. Es utilizada en la investigación de parásitos que pueden
producir enfermedades en plantas y animales con objeto de analizar, diagnosticar y
posteriormente establecer un tratamiento óptimo para poder curarlas y erradicarlas.

Gran parte de los parásitos más dificil de tratar son los que se alojan en el
interior del organismo, lo que puede ingresar por via oral o fluidos y gran parte de
estos pueden alojarse en el sistema digestivo, principalmente en estomago e
intestino.

En el contexto de la medicina, la área de \textbf{Tecnología Médica} en su
especialización de parasitología esta encargada de la realización y analisis de
examenes con la finalidad de diagnosticar amenazas relacionadas a la disciplina.

\subsubsection{Exámenes parasitologicos}

Existen muchos tipos de análisis de laboratorio para diagnosticar enfermedades parasitarias.
El tipo de análisis que solicite el médico se basará en sus signos y síntomas presentados
durante la consulta médica, cualquier otra afección médica que pueda tener y sus
antecedentes de viajes.

El análisis de laboratorio se lleva a cabo con las observaciones de muestras entregadas
al laboratorio por el médico tratante. Estas muestras dependen de la busqueda de los
parásitos sospechados y sus posibles ubicaciones, siendo estas muestras de la forma de
sangre, heces, muestras urogenitales, esputo, aspirados o biopsias. La especificidad de
los examanes puede variar en la capacidad de detectar diferentes especies o realizar
busquedas de manera particular.

Estos examenes se pueden dividir en dos categorias:

\begin{itemize}
    \item Invasivos: la adquisición de la muestra requiere intervención
        quirurgica algún tipo como las biopsias.
    \item No invasivos: la toma de la muestra presenta un método de obtención que no
        involucra una intervención invasiva al paciente como serían muestras de sangre o
        heces.
\end{itemize}

\subsubsection{Procedimiento de exámenes}

Para la realización de los examenes se procede de las siguientes formas \cite{c01}:

\begin{enumerate}
    \item \textbf{Exámenes de muestra de sangre}: la muestra es tintada y analizada por
        goteo grueso y/o fino con un microscopio. El goteo fino es una forma de repartir
        la muestra en un portaobjeto \footnote{Placa de acrílico trasparente usada para
        manejo de muestras para microscopio} a manera de dejar una capa delgada y
        uniforme en la cual realizar observasiones, el goteo grueso por otro lado,
        consiste en soltar una gota de muestra de forma que la tensión superficial de la
        muestra mantenga su forma circular para dejar decantar las celulas contenidas en
        la muestra al fondo.  La tinción es el proceso en el cual se suman compuestos a
        la muestra que reaccionan a componentes conocidos con el fin de teñir componente
        para facilitar la visualización.
    \item \textbf{Endoscopia/Colonoscopia}: Conciste en la insersión en la boca
        (endoscopia) o el recto (colonoscopia) de una sonda con la cual el médico,
        normalmente un gastroenterólogo, para una examinación diracta.
    \item \textbf{Exámenes seriado de deposiciones}: consiste en la observación de tres
        muestras seriadas \footnote{Muestras tomadas con intervalos de tiempo
        equidistantes con objetivo de muestrear sin que se pierdan ciclos larvarios
        evitando excluir avistamientos} por microscopio luego de haber pasado por
        un centrifugado, utilizado para separar la muestra de liquido concervante, y
        posteriormente tintado para facilidad de observación. La observación se realiza por
        goteo fino.
    \item \textbf{Resonancia Magnetica (RM), Tomografía axial computarizada (TAC)}:
        Pruebas ralizadas para buscar enfermedades parasitarias que pueden provocar
        lesiones en los órganos.
\end{enumerate}

%Aprendizaje Automático
    %Vision por computadora
        %Modelos
        %Conjunto de datos
    %Métricas de evaluación
%Estado del Arte
%% Marco Teorico

\section{Identificación del Problema}

\section{Alternativas de solución}\label{alternativas}

\chapter{Fundamentación y Justificación}\label{fundamentacion}
\section{Justificación y Aporte}
Justificar la conveniencia del proyecto desde diversos puntos de vista.

Preguntas clave:
  \begin{itemize}
  \item ¿Para qué sirve el proyecto?
  \item ¿Quiénes se benefician con los resultados?
  \item ¿Ayuda a resolver algún problema práctico?
  \item ¿Contribuye a aumentar el conocimiento?
  \item ¿Se podrán generalizar los resultado?
\end{itemize}


\begin{ejemplo}
\blindtext %reemplazar esta linea
\end{ejemplo}


\section{Objetivos}\label{objetivos}

Se deben abordar desde el principio de la investigación, expresan los fines que se esperan lograr con el estudio del problema planteado, responden a la pregunta \naranjo{¿Para qué se lleva a cabo la investigación?}, por lo general comienzan con un verbo en infinitivo: Determinar, identificar, establecer, distinguir, medir, cuantificar, entre otros.

Deben enunciar un resultado unívoco, preciso, factible y medible. Su formulación debe ser clara, concisa y bien orientada hacia el fin, en función de ellos se plantean los métodos de recolección de datos, pruebas estadísticas, entre otros.

Evitar unir objetivos, idealmente, un objetivo general y varios específicos.

Cada objetivo específico se ``mapea'' a una pregunta de investigación.
Por ejemplo:
  \begin{itemize}
  \item \textbf{\naranjo{Objetivo:}} Optimizar los métodos de acceso a disco.
  \item \textbf{\naranjo{Preguntas de investigación:}} ¿Cuáles son los métodos de acceso a disco?
\end{itemize}
\subsection{General}
\blindtext %reemplazar esta linea

\subsection{Específicos}
\begin{enumerate}\justifying
  \item \blindtext %reemplazar esta linea
 
  \item \blindtext %reemplazar esta linea

\end{enumerate}

\section{Alcance}
Que se planea realizar y hasta que punto se espera llegar.

Esta subdivisión debe:
\begin{enumerate}\justifying
  \item Identifique el producto del software para ser diseñado por el nombre (por ejemplo, Anfitrión DBMS, el Generador del Reporte, etc.);
  \item Explique eso que el producto (del software hará y que no hará.
  \item Describe la aplicación del software especificándose los beneficios pertinentes, objetivos, y metas;
  \item Sea consistente con las declaraciones similares en las especificaciones de niveles superiores (por ejemplo, las especificaciones de los requisitos del sistema), si ellos existen.
\end{enumerate}




\section{Metodología de Trabajo}
Esto no es hacer referencia a métodos y herramientas que se usarán en el desarrollo del trabajo. Sino que describir cómo se llevará a cabo el trabajo.

Por lo tanto, nuevamente se puede plantear la solución (el proyecto) en términos explícitos de: los objetivos generales y específicos.

Posteriormente relacionar el cumplimiento de los objetivos específicos con tareas o actividades a desarrollar (al final se debe incluir seguramente actividades de validación y prueba del producto - plan de prueba).



En la Tabla \ref{t:info} se muestran las características de los sistemas GNU/Linux,
obtenidas desde \cite{c02}.


\begin{table}[hbt]
\begin{center}
\begin{tabular}{|l|p{10cm}|}\hline
\multicolumn{2}{|c|}{\textbf{Información general}}\\
\hline
\textbf{Modelo de desarrollo}&desarrollo	Software libre y código abierto\\
\textbf{Última versión estable}&Kernel: 4.11.3 (info) 25 de mayo de 2017 (10 días)\\
\textbf{Última versión en pruebas}&	4.12.rc2 (info) 22 de mayo de 2017 (13 días)\\
\textbf{Escrito en}&	C\\
\textbf{Núcleo}&	Núcleo Linux\\
\textbf{Plataformas soportadas}	& DEC Alpha, ARM, AVR32, Blackfin, ETRAX CRIS, FR-V, H8/300, Itanium, M32R, m68k, Microblaze, MIPS, MN103, PA-RISC, PowerPC, s390, S+core, SuperH, SPARC, TILE64, Unicore32, x86, Xtensa\\
\textbf{Licencia}	&GNU General Public License y otras\\
\textbf{Estado actual}	&En desarrollo\\
\textbf{En español}	&Sí\\
\hline
\end{tabular}
\end{center}
\caption{Información General de GNU/Linux}
\label{t:info}
\end{table}

\begin{landscape}
\subsection{Carta Gantt}\label{gantt}
\begin{figure}[hbt]
  \centering
  %\includegraphics{}
  \caption{Carta Gantt del Proyecto XYZ}
  \label{gantt}
\end{figure}
\end{landscape}

\subsubsection{Equipo de Trabajo}
Se describe cada miembro del equipo y sus funciones según la carta gantt.

\subsection{Planificación}
Se describen las subfunciones ha realizar para cumplir cada punto de la carta gantt y quien es(son) el responsable de cada punto.

\subsection{Desglose de Actividades}\label{PERT}
En esta sección se describen cada una de las actividades, duración, dependencias, caminos críticos, entre otras y se debe dar una conclusión de lo mismo.
\begin{figure}[hbt]
\begin{tabular}{|c|c|c|c|c|}\hline
  \textbf{Actividad}&\textbf{Duración} &\textbf{Después de} & \textbf{Simultanea} & \textbf{Antes de}\\\hline
& & &&\\\hline

\end{tabular}
  \caption{Duración de tareas y dependencias}
\end{figure}

\begin{landscape}
\begin{figure}[hbt]
  \centering
  %\includegraphics{}
  \caption{Grafo de Actividades del Proyecto XYZ}
  \label{CPM}
\end{figure}
\end{landscape}

\begin{landscape}
\begin{figure}[hbt]
  \centering
  %\includegraphics{}
  \caption{Grafo de Actividades con duración del Proyecto XYZ}
  \label{CPMduracion}
\end{figure}
\end{landscape}

\begin{figure}[hbt]
 \begin{tabular}{|c|c|cc|cc|c|c|}\hline
 & & \multicolumn{2}{|c|}{\textbf{Inicio}} & \multicolumn{2}{|c|}{\textbf{Termino}} & \textbf{Holgura} & \\
\textbf{Actividad}& \textbf{Duración}& \textbf{Temprano} &\textbf{Tardío} &\textbf{Temprano} &\textbf{Tardío} &\textbf{Total}  &\textbf{Crítico} \\\hline
& & &   & &   & & \\\hline

\end{tabular}
  \caption{Cálculo del diagrama de actividades}
\end{figure}

\begin{landscape}
\begin{figure}[hbt]
  \centering
  %\includegraphics{}
  \caption{Grafo de Actividades con duración y caminos críticos}
  \label{CPMcritico}
\end{figure}
\end{landscape}




\chapter{Descripción General (\rojo{OPCIONAL})}\label{descripcion}
\section{Perspectiva del Producto}
\section{Funciones del Sistema}
Acá se describen las partes más relevantes que tendrá el sistema.
\section{Características de los Usuarios}
\section{Restricciones}
\section{Suposiciones y Dependencias}
\subsection{Suposiciones}
\subsection{Dependencias}
\section{Metodología de Desarrollo (\rojo{OBLIGATORIO})}
Explicación completa de la metodología.

\chapter{Análisis Económico (\rojo{OPCIONAL})}\label{economico}

Aquí hay que añadir secciones según materia vista en el curso de formulación y evaluación de proyectos.

\section{Viabilidad}
Analizar la disponibilidad de recursos financieros, humanos y materiales.

Preguntas clave:
  \begin{itemize}\justifying
  \item ¿Puede llevarse a cabo esta investigación?
  \item ¿Cuánto tiempo tomará realizarla?
\end{itemize}

deben añadir análisis económico de carta Gantt (Figura \ref{gantt}) y malla CPM (Figura \ref{CPMcritico})




\chapter{Requisitos del Proyecto}\label{requisitos}
Entre todos los requisitos estos deben ser complementados con los distintos diagramas UML para dar mejor entendimiento.
\section{Requisitos de Interfaces Externos (\rojo{OPCIONAL})}
\subsection{Interfaces de Usuario (\rojo{OPCIONAL})}
\subsection{Interfaces de Hardware (\rojo{OPCIONAL})}
\subsection{Interfaces de Software (\rojo{OPCIONAL})}
\subsection{Interfaces de Comunicación (\rojo{OPCIONAL})}


\subsection{Modelo de Base de Datos (\rojo{Obligatorio})}
Descripción del diseño de la base de datos, sus relaciones y correspondencia, entre otros.

\begin{landscape}
\begin{figure}[hbt]
  \centering
  %\includegraphics{}
  \caption{Modelo Entidad Relación de la Base de Datos}
  \label{MER}
\end{figure}
\end{landscape}

\begin{landscape}
\begin{figure}[hbt]
  \centering
  %\includegraphics{}
  \caption{Modelo Relacional normalizado hasta XX forma normal de la Base de Datos}
  \label{MRnormalizado}
\end{figure}
\end{landscape}

\section{Requisitos Funcionales (\rojo{Obligatorio})}
\subsection{Requisito funcional x}
\subsection{Requisito funcional y}
\section{Requisitos de Rendimiento (\rojo{Opcional})}
\section{Requisitos de Desarrollo (\rojo{Opcional})}
\section{Requisitos de Tecnológicos (\rojo{Obligatorio})}
\section{Seguridad (\rojo{Obligatorio})}

\chapter{Desarrollo}
\section{Lenguaje de Programación Elegido}
\subsection{Propiedades}
\subsection{Ventajas y Desventajas}

\section{Propuesta de Solución}



\chapter{Aseguramiento de la Calidad (\rojo{Opcional})}\label{calidad}
\section{Modelo}
\section{Técnica}
\section{Implementación}

\chapter{Proceso de Prueba del Software (\rojo{Opcional})}\label{prueba}
\section{Prueba de Unidad}
\section{Prueba de Integración}
\subsection{Integración Descendente}
\subsection{Integración Ascendente}
\subsection{Prueba de Regresión}
\section{Prueba de Sistema}
\subsection{Prueba de Recuperación}
\subsection{Prueba de Seguridad}
\subsection{Prueba de Resistencia}
\subsection{Prueba de Rendimiento}
\section{Prueba de Aceptación}
\section{Criterios de Prueba}
\subsection{Enfoque de Prueba de Caja Negra}
\subsection{Enfoque de Prueba de Caja Blanca}



\chapter{Conclusión}\label{conclusion}
En las conclusiones se destaca lo mostrado en el trabajo, resaltando los resultados. Se indican los trabajos futuros. Usualmente, luego de las conclusiones se incluye un párrafo de agradecimientos a quienes auspician la investigación.
\section{Principales aportes}
\section{Contraste de resultados}
%\section{Trabajos futuros} Solo si corresponde.





%%%%%REFERENCIAS
\renewcommand{\refname}{Referencias}

%agregar referencias
\bibliographystyle{apalike}\bibliography{document.bib}

\renewcommand{\appendixname}{Anexos}
\appendix

\chapter{Definciones, Acronimos y Abreviaturas}\label{definiciones}
\section{Definiciones}
\section{Acrónimos}
\section{Abreviaturas}




\chapter{Configuraciones}\label{configuracion}
\blindtext %reemplazar esta linea

\chapter{Anexo de Código}\label{codigoA}
\lstset{language=SQL}
   \vspace{-0.8cm}
\begin{lstlisting}
-- Database: acuario

-- DROP DATABASE acuario;

CREATE DATABASE acuario
  WITH OWNER = postgres;


CREATE TABLE especies(
    sno integer PRIMARY KEY,
    nombre character varying(20),
    alimento character varying(20)
);

CREATE TABLE tanques(
    tno integer PRIMARY KEY,
    nombre_tanque character varying(20),
    color_tanque character varying(20),
    volumen  integer NOT NULL
);

CREATE TABLE peces(
    pno integer PRIMARY KEY,
    nombre_peces character varying(20),
    color_peces character varying(20),
    tno integer NOT NULL,
    sno integer NOT NULL,
    FOREIGN KEY (tno) REFERENCES tanques (tno) ON UPDATE CASCADE ON DELETE CASCADE,
    FOREIGN KEY (sno) REFERENCES especies (sno) ON UPDATE CASCADE ON DELETE CASCADE
);

CREATE TABLE eventos(
    eno integer PRIMARY KEY,
    pno integer NOT NULL,
    fecha date,
    FOREIGN KEY (pno) REFERENCES peces (pno) ON UPDATE CASCADE ON DELETE CASCADE
);



INSERT INTO especies VALUES(17,'delfin','arenque');
INSERT INTO especies VALUES(22,'tiburon','cualquier cosa');
INSERT INTO especies VALUES(74,'olomina','gusano');
INSERT INTO especies VALUES(93,'ballena','mantequilla de mani');
INSERT INTO especies VALUES(100,'pez espada','gusano');
INSERT INTO especies VALUES(120,'pez globo','gusano');

-- select * from especies

INSERT INTO tanques VALUES(55,'charco','verde',300);
INSERT INTO tanques VALUES(42,'letrina','azul',100);
INSERT INTO tanques VALUES(35,'laguna','rojo',400);
INSERT INTO tanques VALUES(85,'letrina','azul',100);
INSERT INTO tanques VALUES(38,'playa','azul',200);
INSERT INTO tanques VALUES(44,'laguna','verde',200);

-- select * from tanques


INSERT INTO peces VALUES (164, 'charlie', 'naranjo', 42, 74);
INSERT INTO peces VALUES (347, 'flipper', 'negro', 35, 17);
INSERT INTO peces VALUES (228, 'killer', 'blanco', 42, 22);
INSERT INTO peces VALUES (281, 'albert', 'rojo', 55, 17);
INSERT INTO peces VALUES (119, 'bonnie', 'azul', 42, 22);
INSERT INTO peces VALUES (388, 'cory', 'morado', 35, 93);
INSERT INTO peces VALUES (700, 'maureen', 'blanco', 44, 100);
INSERT INTO peces VALUES (800, 'beni', 'rojo', 55, 17);
INSERT INTO peces VALUES (900, 'nemo', 'rojo', 44, 74);
INSERT INTO peces VALUES (150, 'vicky', 'rojo', 55, 100);
INSERT INTO peces VALUES (160, 'mati', 'amarillo', 42, 100);
INSERT INTO peces VALUES (110, 'rafa', 'azul', 85, 100);
INSERT INTO peces VALUES (222, 'jimmy', 'amarillo', 38, 100);
INSERT INTO peces VALUES (144, 'bisho', 'rojo', 42, 93);
INSERT INTO peces VALUES (125, 'chris', 'azul', 38, 93);
INSERT INTO peces VALUES (183, 'sable', 'amarillo', 44, 93);
INSERT INTO peces VALUES (241, 'taz', 'rojo', 55, 93);
INSERT INTO peces VALUES (300, 'baltazar', 'azul', 85, 100);
INSERT INTO peces VALUES (200, 'cash', 'azul', 85, 100);
INSERT INTO peces VALUES (424, 'bandido', 'verde', 35, 100);
INSERT INTO peces VALUES (454, 'romo', 'blanco', 85, 93);


-- select * from peces

INSERT INTO eventos VALUES 
(3456 , 347 , '2010-01-26'),
(6653 , 164 , '2010-05-14'),
(5644 , 347 , '2010-05-15'),
(5645 , 347 , '2010-05-30'),
(6789 , 281 , '2010-04-30'),
(5211 , 228 , '2010-08-20'),
(6719 , 700 , '2010-10-22'),
(4555 , 164 , '2011-11-03'),
(9647 , 281 , '2011-12-06'),
(5347 , 281 , '2011-01-01');

--INSERT INTO eventos VALUES (3456, 164, '2010-01-26'); 
--INSERT INTO eventos VALUES (6653, 347, '2010-05-14'); 
--INSERT INTO eventos VALUES (5644, 347, '2010-05-15'); 
--INSERT INTO eventos VALUES (5645, 347, '2010-05-30'); 
--INSERT INTO eventos VALUES (6789, 228, '2010-04-30'); 
--INSERT INTO eventos VALUES (5211, 119, '2010-08-20'); 
--INSERT INTO eventos VALUES (6719, 388, '2010-10-22'); 
--INSERT INTO eventos VALUES (4555, 164, '2011-11-03'); 
--INSERT INTO eventos VALUES (9647, 281, '2011-12-21'); 
--INSERT INTO eventos VALUES (5369, 281, '2011-01-01'); 


-- ALTER TABLE tanques ADD medida character varying(2); 

-- UPDATE tanques SET medida = 'ml';

-- select * from tanques;

-- ALTER TABLE tanques DROP medida;

-- SELECT * FROM especies;
-- SELECT * FROM tanques;
\end{lstlisting}\vspace{-0.3cm}


\section{Algoritmos}\label{A:alg}
\blindtext %reemplazar esta linea

Lorem ipsum dolor sit amet, consectetur adipiscing elit, sed do eiusmod tempor incididunt ut labore et dolore magna aliqua, como en el Algoritmo \ref{CodC}.

\lstset{language=C}
\begin{lstlisting}[caption = C\'odigo en C de una sumatoria, label = CodC]
#include <stdio.h>
#include <stdlib.h>
/* Algoritmo para realizar la sumatoria */
/* S=2+4+6+...+2n */

int main(void){
	int i,s,n;
	
	/* inicializar el valor de la sumatoria en 0 */
	s=0;
	printf("ingrese la cantidad de elementos de la sumatoria=");
	scanf("% d", &n);
	/* Realiza la iteracion n veces, y el indice "i" lo multiplica por */
	/* 2 y lo va sumando a s*/
	for(i=1;i<=n;i++){
		s = s	+ 2*i;
	} 
	printf("el resultado de la sumatoria es=% d\n",s);

	return (0);
}
\end{lstlisting}


Lorem ipsum dolor sit amet, consectetur adipiscing elit, sed do eiusmod, en el Algoritmo \ref{codL} tempor incididunt ut labore et dolore magna aliqua.

\lstset{language=LISP}
\begin{lstlisting}[caption= C\'odigo LISP de una Lista, label = codL]
(define (length x)
    (if (list? x) (length-aux x)
        (error "x no es una lista")))
        
(define (length-aux x)
    (if (null? x) 0 (+1 (length-aux (cdr x)))))
\end{lstlisting}

Lorem ipsum dolor sit amet, consectetur adipiscing elit, sed do eiusmod tempor incididunt ut, en el Algoritmo \ref{codP} labore et dolore magna aliqua.


\lstset{language=PROLOG}
\begin{lstlisting}[caption= C\'odigo PROLOG de un \'arbol geneal\'ogico, label=codP]
% Arbol genealogico version 1.
% padre(A,B) significa que B es el padre de A.

padre(juan,alberto).
padre(luis,alberto).
padre(alberto,leoncio). 
padre(geronimo,leoncio).
padre(luisa,geronimo). 

% Ahora se define las condiciones para que dos individuos sean hermanos hermano(A,B), significa que A es hermano de B.
hermano(A,B) :- 
    padre(A,P), 
    padre(B,P), 
    A \== B.
% Ahora se define el parentesco abuelo-nieto.  nieto(A,B) significa que A es nieto de B.
nieto(A,B) :- 
    padre(A,P), 
    padre(P,B). 
\end{lstlisting}

Lorem ipsum dolor sit amet, consectetur adipiscing elit, sed do eiusmod tempor incididunt ut labore et dolore magna aliqua.

   \lstset{language=java}
\begin{lstlisting}[caption= C\'odigo JAVA de una clase]
class <Nombre>{
   public static void main(String[] args){
      instrucciones;
   }
}
\end{lstlisting}

\chapter{OPCIONALES en el documento FORMATO}\label{opcional}
\textbf{TODOS LOS TEXTOS ESCRITOS EN CADA SECCIÓN SON SOLO REFERENCIALES Y/O DE AYUDA, POR LO QUE NO DEBEN QUEDAR EN EL DOCUMENTO FINAL.}

Todas las secciones y/o capítulos que no se mencionen en este apartado, son obligatorias, entre ellas los Capitulo \ref{formulacion}, \ref{fundamentacion}, \ref{descripcion} \ref{requisitos}, \ref{conclusion}.

Un caso particular pero que igual es obligatorio es la Sección \ref{alternativas} no es opcional si es un producto único y nuevo ya que aquí se debe explicar porque es novedoso y no hay alternativas.

Los Anexos \ref{configuracion} y \ref{codigoA} igualmente son obligatorios.

\textbf{Opcional} solo queda el Anexo \ref{definiciones}.

En el curso de Taller de Ingeniería de Software los alumnos aprenderán los temas para rellenar los Capitulos \ref{economico}, \ref{calidad} y \ref{prueba}.

En el curso Formulación y evaluación de proyectos el alumno aprenderá como complementar la sección \ref{gantt} al igual que la justificación económica de la malla PERT de la sección \ref{PERT}. De igual forma, el alumno tendrá los conocimientos para realizar la justificación económica del Capitulo \ref{economico}.

Lógicamente esta sección hay que eliminarla (Anexo \ref{opcional}).


\end{document}
