\begin{abstract}
    El examen parasitológico seriado de deposiciones es un examen realizado para la
    detección de presencia parasitológica en pacientes. El examen se lleva a cabo con la
    toma de muestras de deposiciones del paciente y posteriormente se hace observación de
    las mismas por medio de microscopio, en la realización de la observación se documenta
    la confirmación y clasificación de presencia parasitológica en orden de dar con un
    tratamiento certero, eficaz y acorde a las detecciones. Que el examen sea seriado
    refiere a las condiciones en las que estas muestras, tres muestras enfrascadas por
    separado, serán tomadas con la finalidad de tener muestras en distintos ciclos
    larvales.

    El uso de la visión por computadora espera demostrar el aumento en la precisión y
    velocidad de la detección y clasificación de parásitos con objetivo de reducir
    incertidumbre y error humano introducido en la observación manual ejercida en el
    proceso de observación del examen parasitológico seriado de deposiciones. La visión
    por computadora es la combinación de tecnologías que permite a las computadoras
    generar inferencia respecto a imágenes estáticas o en movimiento emulando la visión
    humana por medio de un aprendizaje basado en conjuntos de datos utilizados como
    ejemplos iniciales de los cuales el modelo generado ajustará sus parámetros.

    En la actualidad y en vista de los avances ejercidos en el área se reconoce como
    principal desafió la generación de un conjunto de datos de entrenamiento de los
    cuales se conozcan su grado de sesgo en la información que contenga de manera de
    evitar ajustes de un modelo muy débil o extremadamente limitado en sus capacidades
    de detección y clasificación cayendo en lo que es conocido como \textit{Underfiting}
    y \textit{Overfiting}.

%Resume en un (1) párrafo el contenido del informe en un máximo de 350 palabras.
%Debe ser preciso:
%\begin{itemize}\justifying
  %\item Establece el problema
  %\item Dice porqué es interesante
  %\item Señala los logros y desafios
%\end{itemize}
%Un resumen debe ser llamativo, motivador, descriptivo y sin contenido específico. \textbf{No incluye}: citas, referencias, conclusiones, figuras ni tablas.
%
%
\keywords{Parasitólogia, Serializado, Deposiciones, Visión por Computadora, Modelo,
Conjunto de Datos, \textit{Underfiting}, \textit{Overfiting}}
\end{abstract}
