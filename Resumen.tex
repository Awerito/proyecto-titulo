\begin{abstract}
    El examen parasitológico seriado de deposiciones es un examen realizado para la
    detección de presencia parasitológica en pacientes. El examen se lleva a cabo con la
    toma de muestras de deposiciones del paciente y posteriormente se hace observación de
    las mismas por medio de microscopio, en la realización de la observación se documenta
    la confirmación y clasificación de presencia parasitológica en orden de dar con un
    tratamiento certero, eficaz y acorde a las detecciones.

    El uso de la visión por computadora espera demostrar el aumento en la precisión y
    velocidad de la detección y clasificación de parásitos con objetivo de reducir
    incertidumbre y error humano introducido en la observación manual ejercida en el
    proceso de observación del examen parasitológico seriado de deposiciones. La visión
    por computadora es la combinación de tecnologías que permite a las computadoras
    generar inferencia respecto a imágenes estáticas o en movimiento emulando la visión
    humana por medio de un aprendizaje basado en conjuntos de datos utilizados como
    ejemplos iniciales de los cuales el modelo generado ajustará sus parámetros.

    En el presente trabajo, se realizó una búsqueda de conjuntos de datos sobre parásitos,
    se desarrolló un modelo de clasificación de imágenes basados en visión por computadora y
    se probaron diferentes alternativas para mejorar la calidad de la identificación y la 
    clasificación.

    Finalmente, este trabajo aporta un marco experimental sobre un conjunto de datos acotado
    de parásitos, que puede ser liberado a producción, ampliando la cantidad de parásitos a
    identificar.

%Resume en un (1) párrafo el contenido del informe en un máximo de 350 palabras.
%Debe ser preciso:
%\begin{itemize}\justifying
  %\item Establece el problema
  %\item Dice porqué es interesante
  %\item Señala los logros y desafios
%\end{itemize}
%Un resumen debe ser llamativo, motivador, descriptivo y sin contenido específico. \textbf{No incluye}: citas, referencias, conclusiones, figuras ni tablas.
%
%
\keywords{Parasitólogia, Serializado, Deposiciones, Visión por Computadora, Modelo,
Conjunto de Datos, \textit{Underfiting}, \textit{Overfiting}}
\end{abstract}
